\documentclass[12pt]{article}
\usepackage{graphicx}
\usepackage[margin=25mm]{geometry}
\usepackage{amsmath}
\usepackage{amssymb}
\usepackage{biblatex}
\usepackage{booktabs}
\usepackage{float}
\usepackage{tabularx}
\renewcommand{\thesubsection}{(\alph{subsection})}
\begin{document}

% Cover Page
\pagebreak

\begin{titlepage}
    \begin{center}
        \vspace*{\fill}
        Lab 4: Circuits

        Author: Shaaz Feerasta

        CCID: feerasta

        Student ID: 1704756

        Lab Partner(s): Morgann Reinhart

        PHYS 126, LAB HR81

        TA: Nicolas Concha Marroquin

        Date of Lab: February 13, 2025
        \vspace*{\fill}
    \end{center}
\end{titlepage}

\section{Single Resistor}
\subsection{Resistance}
The resistance of the resistor used was 3.3 k$\Omega =$ 3300 $\Omega$

\subsection{Potential difference and discussion of agreement}
When conducting the experiment, our observed potential difference between points B and D was 3.8 V.
Comparing this to our power supply value of 4.0 V, we can see that the readings are very close.
The values may be off due to simple human error or error in the resistors, power supply, etc. 
The errors may also be due to the potential difference within the battery itself (as it does not have zero internal resistance).

\subsection{Potential difference between A and B}
The observed potential difference between points A and B is 0 V.

\subsection{Current}
Our observed current reading is 1.2 mA $=$ 0.0012 A.

\subsection{Power calculated both ways and discussion of any difference}
For the first equation, $P = IV = (0.0012)(3.8) = 0.00456 $ W. For the second equation,
$P = V^2/R = (3.8)^2/3300 \approx 0.00438$ W. Notice that the result from the first equation is slightly
greater than the result from the second equation. This may be due to rounding errors, or even simply human error when observing
the devices for voltage and current. We assume that the second equation is more accurate since the resistance of the resistor
is provided to us without need for specific measurement, whereas we need to measure and observe both current and voltage for the first equation.

\section{Triple Resistor}
\subsection{Reistances}
The resistor between AD was 3.3 k$\Omega$ (3300 $\Omega$).
The resistor between BE was 4.7 k$\Omega$ (4700 $\Omega$).
The resistor between CF was 3.3 k$\Omega$ (3300 $\Omega$).

\subsection{Potential difference between A and D}
The observed potential difference between points A and D is 3.7 V.

\subsection{Potential difference between B and E}
The observed potential difference between points B and E is 3.7 V.

\subsection{Potential difference between C and F}
The observed potential difference between points C and F is 3.7 V.

\subsection{Current}
$I = I_{AD} + I_{BE} + I_{CF} = 2.3 + 1.9 + 2.3 = 6.5$ mA (0.0065 A)

\subsection{Equivalent resistance and explanation}
Using the rule for total resistance for parallel resistors, we know that the equivalent resistance 
$R = \left(\frac{1}{R_{AD}} + \frac{1}{R_{BE}} + \frac{1}{R_{CF}}\right)^{-1} = 
\left(\frac{1}{3300} + \frac{1}{4700} + \frac{1}{3300}\right)^{-1}
\approx 1221.26 \, \Omega$

\section{Theoretical Calculation}
\subsection{Resistance for series}
The rule for reistance for resistors in series is simply to sum them all up, i.e.
$R = \sum^n_{i=1} R_i = \sum_{i=1}^{10} 10 \, \Omega = 100 \, \Omega$

\subsection{Resistance for parallel}
The rule for resistance for resistors in parallel is to sum the inverses and then inverse again, i.e.
$R = \left(\sum_{i=1}^{n} 1/R_i \right)^{-1} = \left(\sum_{i=1}^{10} 1 / (10 \, \Omega)\right)^{-1} = (1)^{-1} = 1 \, \Omega$

\section{Wheatstone Bridge}
\subsection{Explanation of what resistance was used and why}
Through simple yet tedious trial and error, we come to the conclusion that the 2.2 k$\Omega$ was the resistance
that made the voltmeter read exactly 0V.

\subsection{Discussion of what would happen if resistances were doubled}
When doubling the resistances, theoretically (and experimentally) we would not need to change
the resistance X for the voltmeter to read zero. This is due to the rule for parallel resistances, as the fractions would just
reduce themselves, and the series resistances would add to the same thing essentially. We can also think of it
from the formula provided in the assignment, $R_X = \frac{R_C}{R_B}R_A = \frac{2R_C}{2R_B}R_A$.

\section{Voltage Divider}
\subsection{Discussion of relationship}
To calculate the voltage drop, we first need the total current. Since the resistors are in series, total resistance is
$R = R_1 + R_2 = 1 + 10 = 11$ k$\Omega$ (11000 $\Omega$). Then, $V = IR \implies I = V/R = 4/11000 = 3.64 \times 10^{-4}$ A.
\\ \\
\noindent Now we can calculate the voltage drop across the 1 k$\Omega$ resistor: $V = IR = (3.64 \times 10^{-4}) (1000) = 3.64 \times 10^{-1}$ V
\\ \\ 
And the voltage drop across the 10k$\Omega$ resistor: $V = IR = (3.64 \times 10^{-4}) (10000) = 3.64$ V.
\\ \\
Therefore, we can see that since the resistance of the first resistor is 10 times less than the second one, the voltage
drop is also 10 times less than the voltage drop in the second resistor.

\subsection{Discussion of which resistor was used}
First we need to find the current going into the parellel resistors, which is the same as the current going through our first resistor in series (1 k$\Omega$ resistor).
$V = IR \implies I = V/R = 2/1000 = 0.002$ A. Now we also know that $I_{10\, k\Omega} + I_C = 0.002$ A.
So, we can use the fact that $I = V/R$ to demonstrate that:
\begin{align*}
    V_{10\,k\Omega}/R_{10\,k\Omega} + V_C/R_C &= 0.002 \\
    2/10000 + 2/R_C &= 0.002 \text{, since V is the same for all parallel resistors.} \\
    \therefore 2/R_C &= 0.002 - 2/10000 = 0.0018 \\
    \therefore R_C &= 2/0.0018 = 1111.11 \, \Omega\\
    R_C &\approx 1.11 \text{ k}\Omega
\end{align*}
Therefore, the resistor used would be approximately 1.11 k$\Omega$ in resistance.
\renewcommand{\refname}{References and Acknowledgements}
\begin{thebibliography}{9}
    \bibitem{labmanual} 
    Department of Physics. \textit{PHYS 126 Lab Manual}. University of Alberta, 2025.

    \bibitem{person}
    TA assisted with the lab, and provided guidance on the data collection and analysis.

    \bibitem{person}
    Lab partner Morgann Reinhart assisted with the data collection and analysis.
    
\end{thebibliography}

\end{document}
