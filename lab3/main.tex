\documentclass[12pt]{article}
\usepackage{graphicx}
\usepackage[margin=25mm]{geometry}
\usepackage{amsmath}
\usepackage{amssymb}
\usepackage{biblatex}
\usepackage{booktabs}
\usepackage{float}
\usepackage{tabularx}
\begin{document}

% Cover Page
\pagebreak

\begin{titlepage}
    \begin{center}
        \vspace*{\fill}
        Lab 3: Mapping Electric Potential

        Author: Shaaz Feerasta

        CCID: feerasta

        Student ID: 1704756

        Lab Partner(s): Morgann Reinhart

        PHYS 126, LAB HR81

        TA: Nicolas Concha Marroquin

        Date of Lab: February 6, 2025
        \vspace*{\fill}
    \end{center}
\end{titlepage}

\section{Potentials Map}

\begin{figure}[h]
    \centering
    \includegraphics[width=0.75\textwidth]{Graph.png}
    \caption{Image above demonstrates a photograph of the electric potentials mapped onto a piece of graph paper.
            Shows the shape and contour lines in pencil, along with the labelled potentials at various points
            corresponding to the conductive paper in the lab. The black pen lines show the electric field lines, and the
            selected points for the following questions.}
    \label{fig:diagram}
\end{figure}

\section{Strongest Electric Field}

The electric field is likely to be strongest at the edges of the triangles, rather than the corners. More specifically, the center of the edges.
We know that our change in electric potential goes from 4.50 at the edges of the larger triangle, to 0 at the inner triangle.
The only other part that determines our electric field strength is the distance, and the distance between the 4.50 and 0 is the shortest
at the middle of the edges, thus resulting in the largest electric field strength.

\section{Electric Field}

\begin{table}[H]
    \small
    \caption{Collected measurements of voltages surrounding specific points, and the distances between them.
            The calculated electric field is also included in the table.\\}
    \label{aggiungi}
    \hspace{-15mm} % Moves the table left
    \begin{tabular}{ccccc}
    \toprule
    Point: X, Y & Voltage Point 1 (V) & Voltage Point 2 (V) & Distance Between Points (cm) & Electric Field (V/m)\\
    \midrule
    A: 14, 6  & 4.35 & 3.60  & 3.9 & -19.2\\
    B: 11, 10 & 4.23 & 2.13 & 2.8 & -75.0\\
    C: 7, 15  & 4.46 & 3.59 & 3.4 & -23.6\\
    D: 14, 16 & 3.31 & 1.20  & 2.2 & -95.9\\
    E: 20, 15 & 4.46 & 3.59 & 4.1 & -21.2\\
    \bottomrule
    \end{tabular}
\end{table}

\begin{align*}
    E &= - \frac{\Delta V}{\Delta s} = - \frac{V_1 - V_2}{\Delta s} \\
    E &= - \frac{4.35 - 3.60}{3.9 / 100} = -19.2 \text{ V/m}
\end{align*}

\section{Uncertainty}
We assume that $\delta V_1 = \delta V_2 = 0.01$ V, and $\delta (\Delta s) = 1 \text{ mm} = 0.1 \text{ cm} = 0.001\text{ m}$.
From there, we know that:
\begin{align*}
    \delta E &= - \left |\frac{\Delta V}{\delta s}\right | \sqrt{\left(\frac{\delta(\Delta V)}{\Delta V}\right)^2 \left(\frac{\delta (\Delta s)}{\Delta s}\right)^2} \\
    &= - \left |\frac{V_1 - V_2}{\delta s}\right | \sqrt{\left(\frac{\delta(V_1 - V_2)}{\Delta V}\right)^2 \left(\frac{\delta (\Delta s)}{\Delta s}\right)^2} \\
    &= - \left |\frac{V_1 - V_2}{\delta s}\right | \sqrt{\left(\frac{\sqrt{(\delta V_1)^2 + (\delta V_2)^2}}{V_1 - V_2}\right)^2 \left(\frac{\delta (\Delta s)}{\Delta s}\right)^2} \\
    &= - \left |\frac{4.35 - 3.60}{3.9/100}\right | \sqrt{\left(\frac{\sqrt{(0.01)^2 + (0.01)^2}}{4.35 - 3.60}\right)^2 \left(\frac{0.001}{3.9/100}\right)^2} \\
    \delta E &\approx -0.00929792 = -9 \times 10^{-3} \text{ V/m}
\end{align*}

\renewcommand{\refname}{References and Acknowledgements}
\begin{thebibliography}{9}
    \bibitem{labmanual} 
    Department of Physics. \textit{PHYS 126 Lab Manual}. University of Alberta, 2025.

    \bibitem{person}
    TA assisted with the lab, and provided guidance on the data collection and analysis.

    \bibitem{person}
    Lab partner Morgan Reinhart assisted with the data collection and analysis.
    
\end{thebibliography}

\end{document}
