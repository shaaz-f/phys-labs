\documentclass[12pt]{article}
\usepackage{graphicx}
\usepackage[margin=25mm]{geometry}
\usepackage{amsmath}
\usepackage{amssymb}

\begin{document}
% Cover Page
\pagebreak

\begin{titlepage}
    \begin{center}
        \vspace*{\fill}
        Lab 3: Mapping Electric Potential

        Author: Shaaz Feerasta

        CCID: feerasta

        Student ID: 1704756

        Lab Partner(s): Morgann Reinhart

        PHYS 126, LAB HR81

        TA: Nicolas Concha Marroquin

        Date of Lab: February 6, 2025
        \vspace*{\fill}
    \end{center}
\end{titlepage}

\section{Potentials Map}

\begin{figure}[h]
    \centering
    \includegraphics[width=0.75\textwidth]{Graph.png}
    \caption{Image above demonstrates a photograph of the eletric potentials mapped onto a piece of graph paper.
            Shows the shape and contour lines in pencil, along with the labelled potentials at various points
            corresponding to the conductive paper in the lab. The black pen lines show the electric field lines, and the
            selected points for the following questions.}
    \label{fig:diagram}
\end{figure}

\section{Strongest Electric Field}



\section{Electric Field}

\section{Uncertainty}

\end{document}