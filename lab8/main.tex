\documentclass[12pt]{article}
\usepackage{graphicx}
\usepackage[margin=25mm]{geometry}
\usepackage{amsmath}
\usepackage{amssymb}
\usepackage{biblatex}
\usepackage{booktabs}
\usepackage{float}
\usepackage{tabularx}
\renewcommand{\thesubsection}{(\alph{subsection})}
\begin{document}

% Cover Page
\pagebreak
\begin{titlepage}
    \begin{center}
        \vspace*{\fill}
        Lab 8: Radioactivity and Shielding

        Author: Shaaz Feerasta

        CCID: feerasta

        Student ID: 1704756

        Lab Partner(s): Morgann Reinhart

        PHYS 126, LAB HR81

        TA: Nicolas Concha Marroquin

        Date of Lab: March 20, 2025
        \vspace*{\fill}
    \end{center}
\end{titlepage}

\section{Source Choice}
Gamma has more energy than beta, and thats why we want to use a thicker material for gamma because gamma waves can pass through metal.
That's why we use metal plates for cesium and it's gamma waves.
\section{Table}
\begin{table}[H]
    \centering
    \begin{tabular}{cccccc}
        \toprule
        \multicolumn{3}{c}{Strontium-90} & \multicolumn{3}{c}{Cesium-137} \\
        \cmidrule(lr){1-3} \cmidrule(lr){4-6}
        Thickness (cm) & Count Rate I (cpm) & $\ln[I]$ & Thickness (cm) & Count Rate I (cpm) & $\ln[I]$ \\
        \midrule
        0   & 73 & 4.29 & 0.5 & 48 & 3.87 \\
        0.1 & 44 & 3.78 & 0.8 & 49 & 3.89 \\
        0.2 & 51 & 3.93 & 1.0 & 37 & 3.61 \\
        0.3 & 37 & 3.61 & 1.2 & 36 & 3.58 \\
        0.4 & 34 & 3.53 & 1.4 & 41 & 3.71 \\
        0.5 & 34 & 3.53 & 1.6 & 27 & 3.30 \\
        0.6 & 34 & 3.53 & 1.8 & 28 & 3.33 \\
        0.7 & 37 & 3.61 & 2.1 & 37 & 3.61 \\
        0.8 & 34 & 3.53 & 2.4 & 29 & 3.37 \\
        0.9 & 26 & 3.26 & 2.7 & 33 & 3.50 \\
        1   & 38 & 3.64 & 3.0 & 37 & 3.61 \\
        1.1 & 23 & 3.14 & 3.3 & 24 & 3.18 \\
        1.2 & 39 & 3.66 & 3.6 & 20 & 3.00 \\
        1.3 & 32 & 3.47 & 3.9 & 29 & 3.37 \\
        1.4 & 26 & 3.26 & 4.2 & 33 & 3.50 \\
        1.5 & 24 & 3.18 & 4.5 & 26 & 3.26 \\
        1.6 & 29 & 3.37 & 4.8 & 23 & 3.14 \\
        1.7 & 18 & 2.89 & 5.1 & 25 & 3.22 \\
        1.8 & 32 & 3.47 & 5.4 & 22 & 3.09 \\
        1.9 & 28 & 3.33 & 5.7 & 23 & 3.14 \\
        \bottomrule
    \end{tabular}
    \caption{Collected data of Strontium-90 and Cesium-137 along with data used to plot the equations. $I_0$ is the first row of both sections.}
    \label{tab:count_rate_combined}
\end{table}


\section{Linearization}

So I really wanted our $\mu$ as our slope. So I was thinking to just natural log both sides! Kinda like:
\begin{align*}
    I(x) &= I_0 e^{-\mu x} \\
    \underbrace{\ln[I(x)]}_{y} &= \underbrace{-\mu}_{\text{m}} x + \underbrace{\ln[I_0]}_{\text{b}}
\end{align*}

\section{Graph}



\section{Thickness}

\section{Banana Equivalent}

\begin{thebibliography}{9}
    \bibitem{labmanual} 
    Department of Physics. \textit{PHYS 126 Lab Manual}. University of Alberta, 2025.

    \bibitem{person}
    TA assisted with the lab, and provided guidance on the data collection and analysis.

    \bibitem{person}
    Lab partner Morgann Reinhart assisted with the data collection and analysis.
    
\end{thebibliography}

\end{document}
