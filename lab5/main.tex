\documentclass[12pt]{article}
\usepackage{graphicx}
\usepackage[margin=25mm]{geometry}
\usepackage{amsmath}
\usepackage{amssymb}
\usepackage{biblatex}
\usepackage{booktabs}
\usepackage{float}
\usepackage{tabularx}
\renewcommand{\thesubsection}{(\alph{subsection})}
\begin{document}

% Cover Page
\pagebreak

\begin{titlepage}
    \begin{center}
        \vspace*{\fill}
        Lab 5: Resistance

        Author: Shaaz Feerasta

        CCID: feerasta

        Student ID: 1704756

        Lab Partner(s): Morgann Reinhart

        PHYS 126, LAB HR81

        TA: Nicolas Concha Marroquin

        Date of Lab: February 27, 2025
        \vspace*{\fill}
    \end{center}
\end{titlepage}

\section{Data}

\begin{table}[H]
\caption{Data for Red Colour: Measurements of length, area, voltage, current, and resistivity}
\centering
\begin{tabular}{|c|c|c|c|c|}
\hline
Length (m) & Area (m$^2$) & Voltage (V) & Current (A) & Resistivity ($\Omega$) \\ \hline
0.082   & $9.08 \times 10^{-4}$   & 2.0   & 0.0167  & 1.33 \\ \hline
0.059   & $9.08 \times 10^{-4}$   & 1.2   & 0.0167   & 1.11\\ \hline
0.034   & $9.08 \times 10^{-4}$   & 0.6   & 0.0167   & 0.959 \\ \hline
0.034   & $6.61 \times 10^{-4}$   & 0.7   & 0.0460   & 0.296 \\ \hline
0.018   & $6.61 \times 10^{-4}$   & 0.4   & 0.0460   & 0.319 \\ \hline
0.007   & $6.61 \times 10^{-4}$   & 0.2   & 0.0460   & 0.411 \\ \hline
\end{tabular}
\label{tab:red_data}
\end{table}

\begin{table}[H]
\caption{Data for Blue Colour: Measurements of length, area, voltage, current, and resistivity}
\centering
\begin{tabular}{|c|c|c|c|c|}
\hline
Length (m) & Area (m$^2$) & Voltage (V) & Current (A) & Resistivity ($\Omega$) \\ \hline
0.091   & $9.08 \times 10^{-4}$   & 2.4   & 0.0170  & 1.41 \\ \hline
0.059   & $9.08 \times 10^{-4}$   & 1.5   & 0.0170   & 1.36 \\ \hline
0.034   & $9.08 \times 10^{-4}$   & 0.8   & 0.0170   & 1.26 \\ \hline
0.031   & $6.61 \times 10^{-4}$   & 0.7   & 0.0460   & 0.324 \\ \hline
0.017   & $6.61 \times 10^{-4}$   & 0.5   & 0.0460   & 0.423 \\ \hline
0.008   & $6.61 \times 10^{-4}$   & 0.1   & 0.0460   & 0.180 \\ \hline
\end{tabular}
\label{tab:blue_data}
\end{table}
\section{Uncertainty}
As we deform both of the Play-doh, we seem to notice a slight change in the voltage.
It seems to fluctuate around $\pm 0.2$ V from the original voltage. This number of 0.2 seems
to stay consistent throughout area, length, and Play-doh colour.
It does in fact impact the uncertainty in our experiment, as now we have an additional value we are uncertain about,
increasing the error of our final value.
\section{Statistics}
As outlined in section 6.1, we know that our mean $\mu$ is given by the formula:
\[
\mu_{\text{red}} = \frac{1}{N} \sum_{i=1}^{N} x_i = 0.736 \, \Omega
\]
\[
\mu_{\text{blue}} = \frac{1}{N} \sum_{i=1}^{N} x_i = 0.825 \, \Omega
\]
where $N$ is the number of measurements and $x_i$ represents each individual measurement.

We also know that the standard deviation is given by:
\[
\sigma_{\text{red}} = \sqrt{\frac{1}{N} \sum_{i=1}^{N} (x_i - \mu_{\text{red}})^2} = 0.449
\]
\[
\sigma_{\text{blue}} = \sqrt{\frac{1}{N} \sum_{i=1}^{N} (x_i - \mu_{\text{blue}})^2} = 0.573
\]
where N is the number of measurements, $x_i$ represents each individual measurement, and $mu$ represents the means found above.

This also gives us our mean error from the formula below:
\[
\delta \mu_{\text{red}} = \frac{\sigma_{\text{red}}}{\sqrt{N}} = \frac{0.449}{\sqrt{6}} \approx 0.183
\]
\[
\delta \mu_{\text{blue}} = \frac{\sigma_{\text{blue}}}{\sqrt{N}} = \frac{0.573}{\sqrt{6}} \approx 0.234
\]
\section{Comparison}

When straight up using the mathematical wonder of subtraction, we get: 
\[
\Delta \mu = \mu_{\text{red}} - \mu_{\text{blue}} = 0.736 \, \Omega - 0.825 \, \Omega = -0.089 \, \Omega
\]

However, when using our uncertainties, we can propogate the error!
\[
\delta \Delta \mu = \sqrt{(\delta \mu_{\text{red}})^2 + (\delta \mu_{\text{blue}})^2} = \sqrt{(0.183)^2 + (0.234)^2} \approx 0.294
\]

Therefore, the difference in means with propagated error is:
\[
\mu_{\text{red}} - \mu_{\text{blue}} = -0.089 \, \Omega \pm 0.294 \, \Omega
\]

Since our difference in means is actually between one propogated error interval, we can say that there is good agreement between the means. In other words
we can say that the means are pretty similar and there isn't much of a difference.

\renewcommand{\bibname}{5\ \ \References and Acknowledgements}
\begin{thebibliography}{9}
    \bibitem{labmanual} 
    Department of Physics. \textit{PHYS 126 Lab Manual}. University of Alberta, 2025.

    \bibitem{person}
    TA assisted with the lab, and provided guidance on the data collection and analysis.

    \bibitem{person}
    Lab partner Morgann Reinhart assisted with the data collection and analysis.
    
\end{thebibliography}

\end{document}
